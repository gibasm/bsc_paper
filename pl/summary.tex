% !TEX encoding = UTF-8 Unicode 
% !TEX root = praca.tex

\chapter*{Podsumowanie}

\section{Osiągnięte rezultaty}

Podsumowując, w niniejszej pracy udało się zrealizować postawione cele.
Zaprojektowano i stworzono funkcjonalny prototyp urządzenia oraz oprogramowanie sprzętowe.
Stworzono również aplikację na komputery osobiste, która ułatwia korzystanie z urządzenia.
Główne osiągnięcia pracy obejmują:

1. Opracowanie i wykonanie modułu akwizycji sygnału: zaprojektowanie i wykonanie
makiety sprzętowej składającej się ze wzmacniacza analogowego oraz mikrokontrolera,
co umożliwiło efektywną akwizycję sygnału EKG.

2. Rozwój oprogramowania:
zaprojektowano i zaimplementowano filtry cyfrowe, minimalizując tym samym wpływ zakłóceń 
na jakość sygnału. Stworzona aplikacja umożliwiła podgląd sygnału w czasie rzeczywistym
oraz jego zapis do pliku.

3. Testowanie i walidacja systemu:
przeprowadzono testy manualne potwierdzające że urządzenie działa stabilnie i zgodnie
z oczekiwaniami.

\section{Perspektywy rozwoju}

Rozwój projektu urządzenia jak i oprogramowania otwiera możliwości usprawnienia i dodania 
nowych funkcji. Poniżej przedstawiono niektóre z możliwych dróg rozwoju pracy w przyszłości:

1. Dodanie akumulatora (np. litowo-jonowego) oraz modułu \textit{WiFi} umożliwiając 
bezprzewodowe działanie sprzętu: wprowadzenie zasilania bateryjnego wraz z komunikacją
radiową mogłoby przekształcić urządzenie w przenośny, bezprzewodowy system monitorowania
serca pacjenta, w podobny sposób jak robią to współczesne urządzenia nazywane \textit{holterami}.

2. Udoskonalenie interfejsu użytkownika aplikacji: interfejs tekstowy na którym opierają się 
główne części aplikacji (pomijając okienko wykresu w czasie rzeczywistym) 
nie jest intuicyjnym i prostym interfejsem dla przeciętnego użytkownika.
Rozwój bardziej przejrzystego, graficznego środowiska, z użyciem tzw. \textit{widgetów}
może okazać się o wiele łatwiejszy w obsłudze.

\newpage

3. Projekt płytki PCB i obudowy urządzenia: opracowanie dedykowane płytki drukowanej (\textit{PCB})
i ergonomicznej obudowy pozwoli na miniaturyzację urządzenia, zwiększając jego praktyczność
i trwałość. Projekt dedykowanej płytki drukowanej pozwoli również na zmniejszenie podatności 
na zakłócenia zewnętrzne. 

4. Opracowanie oprogramowania i sprzętu zgodnego ze standardami urządzeń medycznych:
współcześnie, urządzenia medyczne muszą spełniać szeregi standardów bezpieczeństwa i jakości.
Każde takie urządzenie jest akceptowane (lub nie) przez odpowiednie komisje, które dopuszczają
produkt do użycia w środowisku medycznym.

5. Klasyfikacja chorób przy pomocy modeli uczenia maszynowego: metody uczenia maszynowego
mogą okazać się przydatne w procesie diagnostycznym i wspomagać badania przesiewowe.
Wśród nich przykładowo sieci neuronowe mogą posłużyć do klasyfikacji występowania 
arytmii w rejestrowanym sygnale w czasie rzeczywistym.
