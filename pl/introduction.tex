% !TEX encoding = UTF-8 Unicode 
% !TEX root = praca.tex

\chapter*{Wprowadzenie}

Akwizycja i przetwarzanie biosygnałów - w tym elektrokardiogramu (\textit{EKG}) - są kluczowymi aspektami 
współczesnej medycyny, szczególnie z uwagi na rosnący odsetek występowania przewlekłych chorób układu krążenia
na całym świecie \cite{Serhani2020}.  
 

\section*{Cel pracy}

Celem pracy było stworzenie sprzętowego oraz programowego systemu akwizycji oraz wstępnego przetwarzania sygnału \textit{EKG} mającego znamiona urządzenia medycznego.
Autor zastrzega, że praca ma wymiar edukacyjny, a jej celem nie było stworzenie narzędzia diagnostycznego spełniającego standardy urządzenia medycznego. 

\section*{Zakres pracy}

Niniejsza praca opisuje realizację odpowiednio dwóch części:

a) sprzętowej (makieta układu elektronicznego),

b) programowej (oprogramowanie części sprzętowej oraz aplikacja na komputery PC).


Część sprzętowa zrealizowana została jako makieta w oparciu o gotowe płytki drukowane z układami elektronicznymi.


Część programową zrealizowano przy użyciu języków \textit{C} i \textit{C++}, 
poza tym do skryptów i narzędzi pomocniczych zastosowano języki \textit{bash} oraz \textit{Python 3}.
