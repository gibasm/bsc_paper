% !TEX encoding = UTF-8 Unicode 
% !TEX root = praca.tex

\chapter{Makieta sprzętowa}

Istotną częścią pracy jest makieta sprzętowa. Została ona zaprojektowana i wykonana w postaci układu dostępnych na rynku płytek drukowanych
(tzw. płytek rozwojowych) połączonych przewodami stosowanymi powszechnie w elektronice w łączeniu prototypowych układów np. na płytkach stykowych. 
W skład makiety sprzętowej wchodzą:

\begin{itemize}

    \item \textit{KA-NUCLEO-F411CE} - płytka rozwojowa z mikrokontrolerem \textit{STM32F411CE} oraz układem programatora \textit{ST-Link/v2.1} \cite{NUCLEO} (rys. \ref{fig:f411_main} i \ref{fig:f411_stlink}),

    \item \textit{SparkFun Single Lead Heart Rate Monitor - AD8232} - płytka ze złączem na elektrody pomiarowe, układem \textit{AD8232} oraz filtrami analogowymi \cite{AD8232BS},

    \item \textit{Elektrody pomiarowe},

\end{itemize}

Makieta sprzętowa wykonana w pracy ma charakter prototypu urządzenia elektronicznego, 
jej elementy połączone są przewodami które można rozłączać i łączyć w dowolny sposób tak aby zmienić konfigurację urządzenia.

\begin{figure}[h!]
    \centering 
    \includegraphics[scale=0.7]{pl/media/hw_blocks_simplified.png}
    \caption{Uproszczony schemat bloków funkcjonalnych makiety sprzętowej}
    \label{fig:hw_hl}
\end{figure}

Do próbkowania i kwantyzacji sygnału z wyprowadzenia oznaczonego \textit{REFOUT} (rys. \ref{fig:ad8232_sch})
zastosowano kanał nr. 1 przetwornika \textit{ADC} tj. wyprowadzenie z oznaczeniem \textit{PA1} mikrokontrolera 
(rys. \ref{fig:f411_main}). Pomiary przetwornika odbywały się względem masy (\textit{GND}) przy napięciu referencyjnym
$U_{ref} = 3,3V$ (maksymalne mierzone napięcie).

W makiecie użyto również zestawu trzech diod (w jednej obudowie) o kolorach czerwonym, zielonym i niebieskim 
(\textit{RGB}, \textit{D2} - rys. \ref{fig:f411_main}) umożliwiających wyświetlanie dowolnej barwy złożonej 
z wymienionych składowych. Kolorowe diody użyte zostały do realizacji podstawowej komunikacji z użytkownikiem,
informując na przykład o tym czy elektrody są prawidłowo podpięte.

\begin{figure}[h!]
    \centering 
    \includegraphics[scale=0.65]{pl/media/ad8232_sch.png}
    \caption{Schemat elektryczny płytki \cite{AD8232BS} zawierającej układ \textit{AD8232}}
    \label{fig:ad8232_sch}
\end{figure}

Wyprowadzenia oznaczone \textit{SDN}, \textit{LO+} i \textit{LO-} (rys. \ref{fig:ad8232_sch}) zostały połączone odpowiednio 
do wyprowadzeń \textit{PB1}, \textit{PB6} i \textit{PB5} mikrokontrolera (rys. \ref{fig:f411_main}) tak jak przedstawiono na
rys. \ref{fig:hw_hl}. Rysunek \ref{fig:hw_hl} przedstawia jedynie najważniejsze, funkcjonalne użyte w zbudowanej 
makiecie sprzętowej w formie uproszczonej. Linie łączące poszczególne bloki, przedstawione w środku kwadratu zaznaczonego 
linią przerywaną symbolizują wewnętrzne magistrale \textit{AHB} (\textit{Advanced High-performance Bus}) i 
\textit{APB} (\textit{Advanced Peripherial Bus}) mikrokontrolera STM32F411CE, natomiast linie pozostałe przedstawiają pojedyncze
linie sygnałowe (zarówno dla sygnałów cyfrowych jak i analogowych).

\newpage

\begin{figure}[h!]
    \centering 
    \includegraphics[scale=0.9]{pl/media/devimg.jpg}
    \caption{Zdjęcie makiety sprzętowej wraz z opisem.}
    \label{fig:devimg}
\end{figure}

Na zdjęciu z rys. \ref{fig:devimg} przedstawiono fizyczną formę makiety sprzętowej.
Płytki zostały umieszczone na przezroczystym płacie szkła akrylowego i przymocowane przy
pomocy śrub montażowych. Natomiast połączenia między płytkami zrealizowana za pomocą
przedstawionych za zdjęciu, przewodów połączeniowych. Płytka z mikrokontrolerem
posiada złącze \textit{mikro USB}, poprzez które odbywa się wgrywanie oprogramowania
oraz komunikacja z aplikacją na komputerze osobistym.

\begin{figure}[h!]
    \centering 
    \includegraphics[scale=0.5]{pl/media/f411_main.png}
    \caption{Schemat elektryczny głównej części płytki \cite{NUCLEO} zawierającej mikrokontroler \textit{STM32F411CE}}
    \label{fig:f411_main}
\end{figure}

\begin{figure}[h!]
    \centering 
    \includegraphics[scale=0.5]{pl/media/f411_stlink.png}
    \caption{Schemat elektryczny części płytki \cite{NUCLEO} zawierającej programator \textit{STLink/v2.1} 
    wraz z urządzeniem \textit{USB CDC}}
    \label{fig:f411_stlink}
\end{figure}

